\chapter{Combinatoral integration of weight systems}
\label{ch:combinatorial-integration_of_weight_systems}

\begin{mdframed}
	See the .tex comments about questions I still have about this section. (some now answered)
\end{mdframed}

% NOTE:
% [Hutchings]: "The concern of this paper is determining when [secondary obstructions] vanish" (p. 4).
% i.e. for braids: secondary obstructions vanish => any singular braid invariant obeying T4T, T1T integrates fully.
% ...
% [Willerton]: First note the following:
%	Properties of Kontsevich integral => Every weight system (W) is the (mth) derivative of a Vassiliev invariant.
%	i.e. Let the Vassiliev invariant be the W of Z (with lower and higher order terms zero). The derivative is W.
%	i.e. This is the Kontsevich theorem / FTVI
% But we can also prove such a fact ``directly'' by integration: by finding all the weight systems and setting them to zero.
% ...
% It would be sensible if the Hutchings implication went both ways. i.e. If any invariant satisfying the primary obstructions
% integrates fully, surely it means the secondary obstructions are zero. What does our partial result say about the secondart obstructions?

\section{Willerton's ``half''-integration}

\section{Setting some more of the constants}

\section{Computational analysis}
