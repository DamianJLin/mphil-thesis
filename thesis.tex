\documentclass[12pt]{report}

%%%%%%%%%%%%%%%%%%%%%%%%%%%%%%%%%%%%%%%%%%%%%%%%%%%%%%%%

%%% General Packages
\usepackage{amsmath, amssymb, amsthm}
\usepackage{titling}
\usepackage{titlesec}
\usepackage{geometry}
\usepackage{enumerate}

\usepackage[hidelinks]{hyperref}



%%% Font and Text Packages
\usepackage{newpxtext}
\usepackage{newpxmath}

\usepackage[dvipsnames]{xcolor}


%%% Graphics, Figure and Listing Packages
\usepackage{graphicx}
\usepackage{float}
\usepackage{calc}
\usepackage{caption}
\usepackage{subcaption}


%%% Table Packages
\usepackage{longtable}


%%% Epigraph
\usepackage{epigraph}


%%% Listings
\usepackage{listings}
\usepackage{lstfiracode}


%%% Bibliography Packages
\usepackage[style=alphabetic]{biblatex}
\bibliography{references}

%%%%%%%%%%%%%%%%%%%%%%%%%%%%%%%%%%%%%%%%%%%%%%%%%%%%%%%%

%%% Page Formatting Options
\geometry{left = 2.5cm}
\geometry{right = 2.5cm}
\geometry{top = 2.5cm}
\geometry{bottom = 2.5cm}

%%% Section and Chapter Titling Options

\titleformat{\chapter}[display]
{\normalfont\bfseries\LARGE}
{\chaptertitlename~\thechapter}
{0pc}
{{\color{black!30!white}\titlerule[2pt]}\vspace{0.8pc}\normalfont\Large}

\titleformat{name=\chapter, numberless}[display]
{\normalfont\bfseries\LARGE}{}{1pc}
{\normalfont\Large}

\titlespacing*{\chapter}{0pt}{30pt}{40pt}

\titleformat{\section}
{\normalfont\bfseries\large}
{\normalfont\bfseries\large{\thesection}}
{1em}
{}

%%% Hyperlink Formatting Options
\hypersetup{
        colorlinks,
        linkcolor={black},
        citecolor={blue!60!black},
        urlcolor={blue!80!black}
}

%%% Epigraph Options and Setup
\setlength\epigraphwidth{0.6\textwidth}
\setlength{\epigraphrule}{0pt}


%%%%%%%%%%%%%%%%%%%%%%%%%%%%%%%%%%%%%%%%%%%%%%%%%%%%%%%%

%%% Graphics and Figure Options
% Graphics path (necessary for .svg images).
\graphicspath{{graphics/}}
\counterwithout{figure}{chapter}
\counterwithout{table}{chapter}

% Caption setup
\captionsetup{margin=1.5cm}


%%% Listings Options
\definecolor{codegreen}{rgb}{0,0.6,0}
\definecolor{codegray}{rgb}{0.5,0.5,0.5}
\definecolor{codepurple}{rgb}{0.58,0,0.82}
\definecolor{codeback}{rgb}{0.95,0.95,0.92}
\lstset{
        language=Python,
        backgroundcolor=\color{codeback},   
        commentstyle=\color{codegreen},
        keywordstyle=\color{magenta},
        numberstyle=\tiny\color{codegray},
        style=FiraCodeStyle,   % Use predefined FiraCodeStyle
        basicstyle=\ttfamily,   % Use \ttfamily for source code listings
        numbers=left
}

%%%%%%%%%%%%%%%%%%%%%%%%%%%%%%%%%%%%%%%%%%%%%%%%%%%%%%%%

%%% Personal Macros
\newcommand{\N}{\mathbb{N}}
\newcommand{\R}{\mathbb{R}}
\newcommand{\Z}{\mathbb{Z}}
\newcommand{\T}{\mathbb{T}}
\renewcommand{\S}{\mathbb{S}}
\newcommand{\ip}[2]{\langle #1, #2 \rangle}
\newcommand{\inject}{\lhook\joinrel\longrightarrow}

\newcommand{\kob}{\operatorname{kob}}
\newcommand{\lk}{\operatorname{\ell\textit{k}}}

\newcommand{\hash}{\ensuremath{\mathbin{\#}}}
\renewcommand{\over}{\text{O}}
\newcommand{\under}{\text{U}}

%%% Drafting Macros
\newcommand{\notered}[1]{{\color{Red} \textbf{#1}}}
\newcommand{\notegreen}[1]{{\color{Green} \textbf{#1}}}
\newcommand{\noteblue}[1]{{\color{Blue} \textbf{#1}}}

%%% amsthm Options
\newtheorem*{theorem}{Theorem}
\newtheorem*{proposition}{Proposition}
\newtheorem*{conjecture}{Conjecture}
\newtheorem*{lemma}{Lemma}
\newtheorem*{corollary}{Corollary}

\newtheoremstyle{upright}{3pt}{3pt}{}{}{\bfseries}{.}{.5em}{}
\theoremstyle{upright}
\newtheorem*{algorithm}{Algorithm}


%%%%%%%%%%%%%%%%%%%%%%%%%%%%%%%%%%%%%%%%%%%%%%%%%%%%%%%%

\begin{document}

        %%% Make titlepage.

        % Titlepage Options
        \author{Damian Lin}
        \title{(Towards) A Unified Topological Kashiwara-Vergne Theory}

        \cleardoublepage \thispagestyle{empty}
        \null \vfil
        \begingroup
        \LARGE \bfseries \centering
        \openup \medskipamount
        \thetitle \par \vspace{30pt}
        \centering \mdseries \theauthor \par \bigskip
        \endgroup
        \vfil \vfil \vfil
        \begin{center}
                An essay submitted in partial fulfilment of\\
                the requirements for the degree of\\
                Master of Philosophy (Science)
                \vfil\vfil
                {\large Pure Mathematics\\[5pt]
                        University of Sydney}\\
                \vskip6mm
                \includegraphics[width=25mm]{graphics/USY_MB1_CMYK_Stacked_Logo}
                \vfil
                \normalsize\today
        \end{center}
        \vfil
        \cleardoublepage

        \tableofcontents

        \chapter*{Introduction}
        \addcontentsline{toc}{chapter}{Introduction}

        This is the introduction.


        \chapter*{Acknowledgements}
        \addcontentsline{toc}{chapter}{Acknowledgements}

        These are the acknowledgements.

        \chapter{Formality}
        Filtered structures, associated graded structures, formality and how this leads to conections between Vassiliev Invariants and quantum algebra via a general application of Von Dyck's Theorem. Can include intro to PaT Vassiliev filatration, chord diagrams, Drinfeld Associators on a story sort of level.

        \section{Filtrations}
        \section{Associated Graded Functor}

        This is a reference to \cite{the-knot-book}.

        \section{Finite Type Invariants and Chord Diagrams}

        \chapter{Lie Theory and the Kashiwara-Vergne Problem}

        Base this on Alekseev-Torossian or even WKOII.

        \chapter{Existing Topological Interpretations of the Kashiwara Vergne Equations}

        \section{Lie Theory}

        \section{Goldman-Turaev}

        \section{Welded Foams}

        \chapter{Topological Approaches}
        Welded foams (WKOII) vs Goldman-Turaev (AKKN) - pointing out the differences.

        Would be good to try writing here.

        \chapter{Emergent Tangles: Lifting Goldman-Turaev to 3 dimensions}
        Zsuzsi et al paper in the works

        \chapter{Emergent w-foams: Lifting goldman-Turaev to 4 dimensions}
        Needs to be done mathematically

        \chapter{Virtual Knot Tabulation?}

        \appendix
        \titleformat{\chapter}[display]
        {\normalfont\bfseries\LARGE}
        {\chaptertitlename~\thechapter}
        {0pc}
        {{\color{black!30!white}\titlerule[0.7pt]}{\color{white}\titlerule[0.7pt]}{\color{black!30!white}\titlerule[0.7pt]}\vspace{0.8pc}\normalfont\Large}

        \chapter{appendixname}
        This is the first appendix. The subject will be bialgebras.


        \newpage
        \printbibliography[title=References]


\end{document}
