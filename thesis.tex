\documentclass[12pt]{report}

%%%%%%%%%%%%%%%%%%%%%%%%%%%%%%%%%%%%%%%%%%%%%%%%%%%%%%%%

%%% General Packages
\usepackage{amsmath, amssymb, amsthm}
\usepackage{titling}
\usepackage{titlesec}
\usepackage{geometry}
\usepackage{enumerate}

\usepackage[hidelinks]{hyperref}



%%% Font and Text Packages
\usepackage{newpxtext}
\usepackage{newpxmath}

\usepackage[dvipsnames]{xcolor}


%%% Graphics, Figure and Listing Packages
\usepackage{graphicx}
\usepackage{xcolor}
\usepackage{float}
\usepackage{calc}
\usepackage{caption}
\usepackage{subcaption}
\usepackage[framemethod=tikz]{mdframed}


%%% Epigraph
\usepackage{epigraph}

%%% Bibliography Packages
\usepackage[style=alphabetic]{biblatex}
\bibliography{references}

%%%%%%%%%%%%%%%%%%%%%%%%%%%%%%%%%%%%%%%%%%%%%%%%%%%%%%%%

%%% Page Formatting Options
\geometry{left = 2.5cm}
\geometry{right = 2.5cm}
\geometry{top = 2.5cm}
\geometry{bottom = 2.5cm}

%%% Section and Chapter Titling Options

\titleformat{\chapter}[display]
{\normalfont\bfseries\LARGE}
{\chaptertitlename~\thechapter}
{0pc}
{{\color{black!30!white}\titlerule[2pt]}\vspace{0.8pc}\normalfont\Large}

\titleformat{name=\chapter, numberless}[display]
{\normalfont\bfseries\LARGE}{}{1pc}
{\normalfont\Large}

\titlespacing*{\chapter}{0pt}{30pt}{40pt}

\titleformat{\section}
{\normalfont\bfseries\large}
{\normalfont\bfseries\large{\thesection}}
{1em}
{}

%%% Hyperlink Formatting Options
\hypersetup{
        colorlinks,
        linkcolor={black},
        citecolor={blue!60!black},
        urlcolor={blue!80!black}
}

%%% Epigraph Options and Setup
\setlength\epigraphwidth{0.6\textwidth}
\setlength{\epigraphrule}{0pt}


%%%%%%%%%%%%%%%%%%%%%%%%%%%%%%%%%%%%%%%%%%%%%%%%%%%%%%%%

%%% Graphics and Figure Options
% Graphics path (necessary for .svg images).
\graphicspath{{graphics/}}
\counterwithout{figure}{chapter}
\counterwithout{table}{chapter}

% Caption setup
\captionsetup{margin=1.5cm}

%%%%%%%%%%%%%%%%%%%%%%%%%%%%%%%%%%%%%%%%%%%%%%%%%%%%%%%%

%%% Palettes

%%%%%%%%%%%%%%%%%%%%%%%%%%%%%%%%%%%%%%%%%%%%%%%%%%%%%%%%


%%% Personal Macros
\newcommand{\N}{\mathbb{N}}
\newcommand{\R}{\mathbb{R}}
\newcommand{\Z}{\mathbb{Z}}
\newcommand{\T}{\mathbb{T}}
\renewcommand{\S}{\mathbb{S}}
\newcommand{\ip}[2]{\langle #1, #2 \rangle}
\newcommand{\inject}{\lhook\joinrel\longrightarrow}

\newcommand{\kob}{\operatorname{kob}}
\newcommand{\lk}{\operatorname{\ell\textit{k}}}

\newcommand{\hash}{\ensuremath{\mathbin{\#}}}
\renewcommand{\over}{\text{O}}
\newcommand{\under}{\text{U}}

%%% amsthm Environments

% Define mdf style
\mdfdefinestyle{lined}{%
        middlelinewidth=2pt,
        middlelinecolor=black,
        bottomline=false,topline=false,rightline=false,
        innertopmargin=-3pt
}

\newtheoremstyle{regular}% name
  {}%         Space above, empty
  {}%         Space below
  {\upshape}% Body font
  {}%         Indent amount (empty = no indent, \parindent = para indent)
  {\bfseries\scshape}% Thm head font
  {.}%        Punctuation after thm head
  {.7em}% Space after thm head: \newline = linebreak
  {}%         Thm head spec

\theoremstyle{regular}

\newtheorem{thm}{Theorem}

\newtheorem{theorem}{Theorem}
\surroundwithmdframed[style=lined,middlelinecolor=PineGreen]{theorem}

\newtheorem{example}{Example}
\surroundwithmdframed[style=lined,middlelinecolor=PineGreen]{example}

\newtheorem{proposition}{Proposition}
\surroundwithmdframed[style=lined,middlelinecolor=Plum]{proposition}

\newtheorem{conjecture}{Conjecture}
\surroundwithmdframed[style=lined,middlelinecolor=BrickRed]{conjecture}

\newtheorem{lemma}{Lemma}
\surroundwithmdframed[style=lined,middlelinecolor=Aquamarine]{lemma}

\newtheorem{corollary}{Corollary}
\surroundwithmdframed[style=lined,middlelinecolor=RoyalPurple]{corollary}

%%% Drafting Macros

\mdfdefinestyle{draftnote}{%
        outerlinewidth=0.4pt,
        innerlinewidth=0.4pt,
        middlelinewidth=1pt,
        middlelinecolor=white,
        backgroundcolor=red!10,
}
\newcommand{\draftnote}[1]{
\begin{mdframed}[style=draftnote]
        {\color{Gray}{\scshape Note:} #1 }
\end{mdframed}
}


%%%%%%%%%%%%%%%%%%%%%%%%%%%%%%%%%%%%%%%%%%%%%%%%%%%%%%%%

\begin{document}

        %%% Make titlepage.

        % Titlepage Options
        \author{Damian Lin}
        \title{(Towards) A Unified Topological Kashiwara-Vergne Theory}

        \cleardoublepage \thispagestyle{empty}
        \null \vfil
        \begingroup
        \LARGE \bfseries \centering
        \openup \medskipamount
        \thetitle \par \vspace{30pt}
        \centering \mdseries \theauthor \par \bigskip
        \endgroup
        \vfil \vfil \vfil
        \begin{center}
                An essay submitted in partial fulfilment of\\
                the requirements for the degree of\\
                Master of Philosophy (Science)
                \vfil\vfil
                {\large Pure Mathematics\\[5pt]
                        University of Sydney}\\
                \vskip6mm
                \includegraphics[width=25mm]{graphics/USY_MB1_CMYK_Stacked_Logo.pdf}
                \vfil
                \normalsize\today
        \end{center}
        \vfil
        \cleardoublepage

        \tableofcontents

        \chapter*{Acknowledgements}
        \addcontentsline{toc}{chapter}{Acknowledgements}

        These are the acknowledgements.

        \chapter*{Introduction}
        \addcontentsline{toc}{chapter}{Introduction}

        \section*{The idea of Finite Type invariants}
        \draftnote{Here I want to motivate the study of Finite-type invariants in general, using analogies from discriminant theory.}

        Various geometric or topological objects can be studied by considering not only those objects, but also their singular versions, and using this `discriminant set' to define and compute invariants. Take for example the following (simple) analogy:

        \begin{example}[Finite type invariants of Quadratic Equations]
                aksldjhkalsjd aksjd lkajs dlkasj lkdaj dklasj dlkajslkdjkashg fkjfsh kjshks
                aksldjhkalsjd aksjd lkajs dlkasj lkdaj dklasj dlkajslkdjkashg fkjfsh kjshks
                aksldjhkalsjd aksjd lkajs dlkasj lkdaj dklasj dlkajslkdjkashg fkjfsh kjshks
                aksldjhkalsjd aksjd lkajs dlkasj lkdaj dklasj dlkajslkdjkashg fkjfsh kjshks
                aksldjhkalsjd aksjd lkajs dlkasj lkdaj dklasj dlkajslkdjkashg fkjfsh kjshks
        \end{example}

        Countless other geometric and topological objects and their invariants can be described this way, for example, the inertial index of a quadratic form, the number of roots of cubic equations, etc. \cite{knots-links-and-their-invariants}.

        \draftnote{Then I want to talk about the Dror Bar-Natan multiavriable calculus analogy a bit here.}

        \chapter{Formality}
        Filtered structures, associated graded structures, formality and how this leads to conections between Vassiliev Invariants and quantum algebra via a general application of Von Dyck's Theorem. Can include intro to PaT Vassiliev filatration, chord diagrams, Drinfeld Associators on a story sort of level.

        \section{Filtrations}
        \section{Associated Graded Functor}

        This is a reference to \cite{the-knot-book}.

        \section{Finite Type Invariants and Chord Diagrams}

        \chapter{Lie Theory and the Kashiwara-Vergne Problem}

        Base this on Alekseev-Torossian or even WKOII.

        \chapter{Existing Topological Interpretations of the Kashiwara Vergne Equations}

        \section{Lie Theory}

        \section{Goldman-Turaev}

        \section{Welded Foams}

        \chapter{Topological Approaches}
        Welded foams (WKOII) vs Goldman-Turaev (AKKN) - pointing out the differences.

        Would be good to try writing here.

        \chapter{Emergent Tangles: Lifting Goldman-Turaev to 3 dimensions}
        Zsuzsi et al paper in the works

        \chapter{Emergent w-foams: Lifting goldman-Turaev to 4 dimensions}
        Needs to be done mathematically

        \chapter{Virtual Knot Tabulation?}

        \appendix
        \titleformat{\chapter}[display]
        {\normalfont\bfseries\LARGE}
        {\chaptertitlename~\thechapter}
        {0pc}
        {{\color{black!30!white}\titlerule[0.7pt]}{\color{white}\titlerule[0.7pt]}{\color{black!30!white}\titlerule[0.7pt]}\vspace{0.8pc}\normalfont\Large}

        \chapter{appendixname}
        This is the first appendix. The subject will be bialgebras.


        \newpage
        \printbibliography[title=References]


\end{document}
