\chapter{Welded knots, isometry Lie algebras and arrow diagrams}
\label{ch:welded-knots-isometry-lie-algebras-and-arrow-diagrams}

Welded knots, or \(w\)-knots are a generalisation of knots first introduced by [who?].

On one hand, topologically, welded knots are more complicated objects, relating to ribbon embeddings of tori in four dimensions. On the other hand, their finite type invariant theory is much more manageable than the classical case, so studying the simpler, welded version of \(\mathcal{A}\) may shed light on how to study the algebra \(\mathcal{A}\).

\section{Welded knots}

A good and thorough exposition of the theory of welded knots is \cite{finite-type-invariants-of-w-knotted-objects-I-w-knots-and-the-alexander-polynomial, finite-type-invariants-of-w-knotted-objects-II-tangles-foams-and-the-kashiwara-vergne-problem}. Their presentation is in terms of virtual knots and circuit algebras.

With classical knots, the definition is first and foremost topological; they are ambient isotopy classes of embeddings of oriented circles into \(\mathbb{R}^{3}\). It is due to the Reidemeister theorem that we may instead work with equivalence classes of planar knot diagrams under the Reidemeister moves. With welded knots, there is a similar topological interpretation, thought its details remain unclear. Welded knot diagrams relate to ambient isotopy classes of ribbon-embedded tori in \(\mathbb{R}^{3}\). However, there are two problems with this classical approach of defining the knot as the ambient isotopy class of the relevant topological objects.

\begin{shaded}
	Firstly, for welded knots, there is a global orientation reversal contianed in the map from the diagrammatic object to the topological object.
\end{shaded}
To avoid this trouble, we choose to work instead with welded long knots. In the classical case, a long knot is the same as a knot, but instead of being \(S^{1} \hookrightarrow \mathbb{R}^{3}\), the embedding is \(k: \mathbb{R} \hookrightarrow \mathbb{R}^{3}\), with the condition that \(k(t) = (t, 0, 0)\) for all large \(|t|\).

For the finite type theory, the algebra \(\mathcal{A}\) of Jacobi diagrams (and chord diagrams), is replaced with \(\mathcal{A}(|)\). The relations are the exact same (either \ref{eq:IHX} or \ref{eq:4T}), but instead of the univalent vertices having a cyclic order (and being drawn on a circle, they have a linear order and are drawn on a line.

Secondly, the relation between welded long knots and long ribbon-knotted tori in \(\mathbb{R}^{4}\) is only conjectural. There is a tube map from welded long knots to long ribbon-knotted tori in \(\mathbb{R}^{4}\) that is known to be surjective but its kernel remains unknown.

The result is a definition of welded knots via diagrams, though it is widely believed that either that welded knots correspond either to long ribbon-knotted tori, or to some small quotient thereof by something akin to framing or rotation number.

\begin{definition}
	A \textbf{welded knot} is an equivalence class of four-valent planar diagram, where each four-valent planar vertex is decorated by a vertex of one of the following types, or rotated versions thereof.
	\[\poscross\;, \quad \negcross\;, \quad \virtcross\]
	The third crossing type is called a \textbf{virtual crossing}. Along each edge of the diagram, the orientations given by the adjacent vertices agree, and such that when opposite planar vertices are considered connected, there is a single connected component.

	The equivalence relations are planar isotopy and the following moves.
	\begin{shaded}
		\begin{itemize}
			\item Reidemeister moves
			\item Virtual reidemeister moves
			\item Overcrossings commute
		\end{itemize}
		\begin{remark}
			Note that we do not allow moves of the form undercrossings commute. Allowing such moves would trivialise welded knots.
		\end{remark}
	\end{shaded}
\end{definition}

\begin{shaded}
	\begin{definition}
		A \textbf{welded long knot} is ...
	\end{definition}
\end{shaded}

\section{Arrow diagrams}

The finite type theory of welded knots is similar to the theory

\section{A universal welded weight system}

section here
