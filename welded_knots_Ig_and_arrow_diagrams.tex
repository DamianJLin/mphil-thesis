\chapter{Welded knots, isometry Lie algebras and arrow diagrams}
\label{ch:welded-knots-isometry-lie-algebras-and-arrow-diagrams}

Welded knots, or \(w\)-knots are a generalisation of knots first introduced by [who?].

On one hand, topologically, welded knots are more complicated objects, relating to ribbon embeddings of tori in four dimensions. On the other hand, their finite type invariant theory is much more manageable than the classical case, so studying the simpler, welded version of \(\mathcal{A}\) may shed light on how to study the algebra \(\mathcal{A}\).

\section{Welded knots}

A good and thorough exposition of the theory of welded knots is \cite{finite-type-invariants-of-w-knotted-objects-I-w-knots-and-the-alexander-polynomial, finite-type-invariants-of-w-knotted-objects-II-tangles-foams-and-the-kashiwara-vergne-problem}. Their presentation is in terms of virtual knots and circuit algebras.

With classical knots, the definition is first and foremost topological; they are ambient isotopy classes of embeddings of oriented circles into \(\mathbb{R}^{3}\). It is due to the Reidemeister theorem that we may instead work with equivalence classes of planar knot diagrams under the Reidemeister moves. With welded knots, there is a similar topological interpretation from \cite{virtual-knot-presentation-of-ribbon-torus-knots}, thought its details remain unclear. Welded knot diagrams relate to ambient isotopy classes of ribbon-embedded tori in \(\mathbb{R}^{3}\). However, there are two problems with this classical approach of defining the knot as the ambient isotopy class of the relevant topological objects.

\begin{shaded}
	Firstly, for welded knots, there is a global orientation reversal contianed in the map from the diagrammatic object to the topological object.
\end{shaded}
To avoid this trouble, we choose to work instead with welded long knots. In the classical case, a long knot is the same as a knot, but instead of being \(S^{1} \hookrightarrow \mathbb{R}^{3}\), the embedding is \(k: \mathbb{R} \hookrightarrow \mathbb{R}^{3}\), with the condition that \(k(t) = (t, 0, 0)\) for all large \(|t|\).

For the finite type theory, the algebra \(\mathcal{A}\) of Jacobi diagrams (and chord diagrams), is replaced with \(\mathcal{A}(|)\). The relations are the exact same (either \ref{eq:IHX} or \ref{eq:4T}), but instead of the univalent vertices having a cyclic order (and being drawn on a circle, they have a linear order and are drawn on a line.

Secondly, the relation between welded long knots and long ribbon-knotted tori in \(\mathbb{R}^{4}\) is only conjectural. There is a tube map from welded long knots to long ribbon-knotted tori in \(\mathbb{R}^{4}\) that is known to be surjective but its kernel remains unknown. A good exposition is \cite{on-the-welded-tube-map}.

The result is a definition of welded knots via diagrams, though it is widely believed that either that welded knots correspond either to long ribbon-knotted tori, or to some small quotient thereof by something akin to framing or rotation number.

\begin{definition}
	A \textbf{welded knot} is an equivalence class of four-valent planar diagram, where each four-valent planar vertex is decorated by a vertex of one of the following types, or rotated versions thereof.
	\[\poscross\;, \quad \negcross\;, \quad \virtcross\]
	The third crossing type is called a \textbf{virtual crossing}. Along each edge of the diagram, the orientations given by the adjacent vertices agree, and such that when opposite planar vertices are considered connected, there is a single connected component.

	The equivalence relations are planar isotopy and the following moves.
	\begin{shaded}
		\begin{itemize}
			\item Reidemeister moves
			\item Virtual reidemeister moves
			\item Overcrossings commute
		\end{itemize}
		\begin{remark}
			Note that we do not allow moves of the form undercrossings commute. Allowing such moves would trivialise welded knots.
		\end{remark}
	\end{shaded}
\end{definition}

\begin{shaded}
	\begin{definition}
		A \textbf{welded long knot} is ...
	\end{definition}
\end{shaded}

\section{Arrow diagrams}

The finite type theory of welded long knots is similar to that of classical long knots.

\begin{definition}
	A \textbf{singular welded long knot} is a welded long knot whose crossings can also be of the following types
	\[\semivirtposcross \;, \quad \semivirtnegcross.\]
	They are known as \textbf{positive semi-virtual crossings} and \textbf{negative semi-virtual crossings}, and they correspond to topological singularities of ribbon-knotted tori.

	% FIX: There's probably better notation.
	The set of \(m\)-singular welded long knots forms an algebra with the product given by connected sum given by concatenation and coproduct \(\Delta(k) = k \otimes k\). This is called the \textbf{algebra of \(m\)-singular welded long knots}, \(\mathcal{K}_{w, m}^{\bullet}(|)\).

	The algebra of \(0\)-singular welded long knots, or just the algebra of welded long knots \(\mathcal{K}^{\bullet}_{w, 0}(|) = \mathcal{K}_{w}(|)\) is filtered by the images of succesive powers of \(\partial\), just like in the classical case, and the filtration is compatible with the bialgebra structure.
\end{definition}

Just as in the classical case, the notions of invariants of \(m\)-singular welded long knots, their differentiability, and the derivative and boundary operators transfer over the the welded case. Here, the derivative and boundary depend on the type of singularity.

\begin{align*}
	\partial \left( \semivirtposcross \right) &= \poscross  -  \virtcross \\
	\partial \left( \semivirtnegcross \right) &= \virtcross - \negcross
\end{align*}

And correspondingly for \(m\)-singular invariants and the operator \(\delta\). As a consequence of these definitons, we can express the positive-to-negative crossing change via a sum of semivirtuals:
\[\partial \left( \semivirtposcross + \semivirtnegcross \right) = \poscross - \negcross.\]

\begin{definitions}
	\begin{enumerate}
		\item A \textbf{long arrow diagram} of order \(m\) is an oriented line with a distinguished set of \(m\) (ordered) pairs of points, considered up to orientation-preserving diffeomorphism of the line. The set of all long arrow diagrams forms an algebra \(\mathcal{D}_{w}(|)\). The product is induced by the natural connected sum operation on long knots, and the coproduct is given by the same formula as in the classical case. This is called the \textbf{bialgebra of long arrow diagrams}, and it is also graded by degree.
		\item The \textbf{long arrow diagram of an \(m\)-singular welded long knot}, denoted \(\sigma(k)\) is the arrow diagram formed as follows. Traverse the welded long knot, and whenever a singularity is encountered, place a point on the oriented line. If the hollow part of the singularity is traversed, place a tail, and if the filled part is traversed place a head. Points corresponding to the same singularity are paired, and the order of the pair is tail first, then head.
	\end{enumerate}
\end{definitions}

\begin{example}
	\begin{shaded}
		Example of arrow diagram and \(\sigma\).
	\end{shaded}
\end{example}

Just as in the classical case, there are natural relations to consider on arrow diagrams. The operations and maps defined above factor through the quotient, and we obtain the algebra \(\mathcal{A}_{w}(|)\). We also refer to arrow diagrams as elements of this algebra. However, we are more interested in the algebra \(\mathcal{J}_{w}(|)\).

\begin{definition}
	A \textbf{long arrow Jacobi diagram} is an oriented unitrivalent graph where every trivalent vertex has at least one vertex oriented inward, and at least one oriented outwards, with the following additional data:
	\begin{itemize}
		\item at each trivalent vertex, a cyclic order of the incident edges,
		\item a fixed linear order on the univalent vertices.
	\end{itemize}
	modulo the following relations:
	\begin{enumerate}
		\begin{shaded}
			\item any directed \textbf{STU} relations
			\item any directed \textbf{AS} relations
			\item any directed \textbf{IHX} relations
			\item the \textbf{TC relation} (tails commute relation)
				\begin{equation}
					\label{eq:TC}
					\tag{\tc}
					\text{tails commuting}
				\end{equation}
		\end{shaded}
	\end{enumerate}
\end{definition}

The main difference between \(\mathcal{J}_{w}(|)\) and \(\mathcal{J}(|)\) is the tails commute relation. This relation is a consequence of the overcrossings commute relation (in a sense it's the shadow of that relation on the associated graded side). This new relation is very important and dramatically reduces the complexity of the algebra.

\begin{proposition}[Two-in-one-out rule]
	In \(\mathcal{J}_{w}(|)\), any diagram with one vertex oriented inwards and two vertices oriented outward vanishes.
\end{proposition}

\begin{shaded}
	\begin{proof}
		Follow the path argument of the classical case. Then apply \ref{eq:TC}.
	\end{proof}
\end{shaded}

Resultingly, only two versions of \ref{eq:STU}, one of \ref{eq:AS} and one of \ref{eq:IHX} remain nontrivial relations in \(\mathcal{J}_{w}(|)\).

\begin{warning}
	From this point it is clear when talking about welded objects we mean long welded objects. So, we write \(\mathcal{A}_{w}(|)\) as \(\mathcal{A}_{w}\), and don't necessarily specify that they are long.
\end{warning}

The existence of a finite type invariant \(Z\) for welded knots establishes a welded version of the fundamental theorem.

\begin{theorem}
	There exists a universal Vassiliev invariant \(Z_{w} : \mathcal{K}_{w} \to \mathcal{A}_{w}\).
\end{theorem}

Unlike the incredible involved construction for \(Z\), the construction of \(Z_{w}\) is relatively simple. The \ref{eq:TC} relation does all of the heavy lifting. A proof can be found in \cite{finite-type-invariants-of-w-knotted-objects-I-w-knots-and-the-alexander-polynomial}.

\begin{corollary}
	We have
	\[\mathcal{A}_{w} \cong \gr \mathcal{A}_{w}.\]
	Equivalently,
	\[\mathcal{V}_{w} \cong \gr \mathcal{W}_{w}.\]
\end{corollary}
\section{Weight systems and Lie algebras}

There is a variant of Construction \ref{cons:lie-algebra-weight-system} for welded knots taking a Lie algebra and constructing a weight system. The difference is that vertices in string diagrams now have incoming edges (``heads'') and outgoing edges (``tails''), and diagramatically, only edges of the opposite orientation can be contracted. This was not true in the classical case, and the need to fix this is what made us restrict to metric Lie algebras. Hence, in the welded case we can consider any finite-dimensional Lie algebra.

Note an imporant difference to the classical case. There are now two types of univalent vertices that can connect to the external line: heads and tails. Heads correspond to elements of \(\mathfrak{g}\) and tails to elements of \(\mathfrak{g}^{\ast}\). Therefore, the role of \(\mathcal{U}(\mathfrak{g})\) in the classical case must be played here by some other associative algebra whose basis elements are words in \(\mathfrak{g} \sqcup \mathfrak{g}^{\ast}\).

Analysing the relations involving the external line should tell us what kind of object replaces \(\mathcal{U}(\mathfrak{g})\).

There are four diagrams of the type
\begin{shaded}
	\[\text{bracket style vertex from } a \otimes b \to c\]
\end{shaded}
Recall that since \ref{eq:STU}s come from \ref{eq:4T}s, that at most two edges may be heads and at most two may be tails around each vertex. Thus there are four cases. In each of the four cases, we have
\begin{shaded}
	\[\text{bracket style vertex from \(a \otimes b \to c\)} = \lambda (a b - b a).\]
\end{shaded}
Of course, in the \ref{eq:TC} case, this is only true becasue \(\lambda = 0\). So, the correct associative algebra target for the weight system is still the Universal enveloping algebra of some Lie algebra, and the Lie algebra as a vector space is \(\mathfrak{g}^{\ast} \oplus \mathfrak{g}\). From the relations above we can determine the structure of this Lie algebra.

The \ref{eq:STU} with two heads on the external line implies that the bracket on \(\mathfrak{g} \oplus 0\) should be inherited from the bracket on \(\mathfrak{g}\).
\begin{shaded}
	Diagram of the regular \ref{eq:STU} with tails marked \(x^{\ast}, y^{\ast}\), and heads on the external line marked \(x, y\), then \([x, y]\).
\end{shaded}

The \ref{eq:TC} relation implies that the bracket on \(0 \oplus \mathfrak{g}\) should be the zero-bracket.
\begin{shaded}
	Diagram of the regular \ref{eq:STU} with tails marked \(x^{\ast}, y^{\ast}\), and heads on the external line marked \(x, y\), then \([x, y]\).
\end{shaded}

Th bracket structure is known on the \(0 \oplus \mathfrak{g}\) and \(\mathfrak{g}^{\ast} \oplus \mathfrak{g}\) direct summands, so the total Lie algebra structure will be a semidirect product \(\mathfrak{g}^{\ast} \rtimes \mathfrak{g}\), and what remains is to specify an action of \(\mathfrak{g}\) on \(\mathfrak{g}^{\ast}\). The remaining `mixed' \ref{eq:STU} relations determine this.
\begin{shaded}
	Diagram of the mixed \ref{eq:STU} with TU external line markings \(x, \phi\) and \(\phi, x\), and markings above \(x^{\ast}\) and \(\phi^{\ast}\). In the S diagram, the external line marking should be the variable \(\theta\).
\end{shaded}
This is to say, if \([\cdot, \cdot] \in \mathfrak{g}^{\ast} \otimes \mathfrak{g}^{\ast} \otimes \mathfrak{g}\), then \(\theta\) satisfies the equation \([\cdot, \cdot] = x^{\ast} \otimes \theta \otimes \phi^{\ast}\), whereby
\[[x, \theta^{\ast}] = \phi^{\ast} \quad \text{ so } \quad \phi([x, \theta^{\ast}]) = 1.\]
So, \(\theta \in \mathfrak{g}^{\ast}\) is the functional
\[\theta: t \longmapsto \phi([x, t]).\]
This is also known as the \textbf{coadjoint action}, \(\theta = \operatorname{ad}^{\ast}_{x}(\phi)\).
The corresponding calculation for the mixed \ref{eq:STU} relation yields \([\psi \oplus 0, 0 \oplus y] = -\operatorname{ad}^{\ast}_{y}(\psi)\).

\begin{definition}
	Let \(\mathfrak{g}\) be a finite-dimensional Lie algebra. Then let \(I\mathfrak{g}\) denote the Lie algebra \(\mathfrak{g}^{\ast} \rtimes \mathfrak{g}\) where \(\mathfrak{g}^{\ast}\) is the abelian Lie algebra on the vector space \(\mathfrak{g}^{\ast}\). The bracket on \(I\mathfrak{g}\) is given by
	\[[\phi \oplus x, \psi \oplus y] = \operatorname{ad}_{x}(\phi) - \operatorname{ad}_{y}(\psi) \oplus [x, y].\]
\end{definition}
\begin{remark}
	This construction coincides with the Drinfeld double of the Lie bialgebra \(\mathfrak{g}\), in the case where \(\mathfrak{g}\) is co-commutative \cite{quantum-groups}.
\end{remark}

\begin{proposition}
	There is a well-defined algebra homomorphism \(\mathcal{A}_{w} \to \mathcal{U}(I\mathfrak{g})\) given by the variant of Construction \ref{cons:lie-algebra-weight-system}.
\end{proposition}

\begin{proof}
	The relations in \(\mathcal{A}_{w}\) are generated by \ref{eq:STU} and \ref{eq:TC}. We have shown that these relations are also true in \(\mathcal{U}(I\mathfrak{g})\).
\end{proof}

\section{The universal welded weight system}

The aim of this section is to prove a Hinnich-Vaintrob style statement for \(\mathcal{A}_{w}\).

\begin{shaded}
	\begin{definition}
		A \textbf{coloured prop} is ...
	\end{definition}

	\begin{definition}
		% TODO: Choose better notation for A ast (it it not necessarily the dual yet).
		The \(2\)-coloured prop \(I\) is the prop generated by \(A\) and \(A^{\ast}\), and the morphisms \(c\), the formal relations from the bracket in \(\mathfrak{g}\), the formal relations from the other bracket.
	\end{definition}

	\begin{proposition}
		The object \(A \otimes A^{\ast}\) is a Lie algebra object in \(I\).
	\end{proposition}
	\begin{proof}

	\end{proof}
\end{shaded}

\begin{shaded}
	\begin{proposition}
		The following associative algebra exists: The external enveloping algebra \(U_{Lie}(A \otimes A)\).
	\end{proposition}
	\begin{proof}
		Follow Hinich and Vaintrob.
	\end{proof}
\end{shaded}

\begin{shaded}
	\begin{theorem}
		The algebra \(A_{w}\) of arrow diagrams is isomorphic (as a bialgebra) to \(U_{Lie}(A \otimes A)\).
	\end{theorem}

	\begin{proof}
		We need to define the metric version of \(I\), maybe call it \(I_{m}\). This will have bubbles.

		Effectively, use a similar argument to what's in Hinnich and Vaintrob.
	\end{proof}
\end{shaded}

